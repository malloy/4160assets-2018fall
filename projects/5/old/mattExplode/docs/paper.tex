\documentclass[times, 11pt]{article}
\usepackage{times}
\usepackage{txfonts}

\newcommand{\cpp}{$C^{++}$}
\ifx\pdfoutput\undefined
  \usepackage[dvips]{graphicx}
  \usepackage[dvips,dvipsnames]{color}
\else
  \usepackage[pdftex]{graphicx}
  \usepackage[pdftex,dvipsnames]{color}

  \pdfinfo {
	/Author (Brian A. Malloy)
	/Title (Milestone: 870)
}
\fi

\def\figline{\rule{\textwidth}{0.3mm}}
\def\shortline{\hrulefill}


\newenvironment{packedEnum}{
\begin{enumerate}
  \setlength{\itemsep}{1pt}
  \setlength{\parskip}{0pt}
  \setlength{\parsep}{0pt}
}{\end{enumerate}}

\newenvironment{packedItem}{
\begin{itemize}
  \setlength{\itemsep}{1pt}
  \setlength{\parskip}{0pt}
  \setlength{\parsep}{0pt}
}{\end{itemize}}

\begin{document}


%don't want date printed
\date{}


\begin{center}
CpSc 870: Object Oriented Programming \\
Computer Science Department \\
Clemson University \\
{\bf Explosions} \\
{\footnotesize Brian Malloy, PhD} \\
\today
\end{center}

\vspace{.2in}

\bibliographystyle{plain}


\date{}
\baselineskip = 16pt
\pagenumbering{arabic}

\begin{figure*}[t]
\centerline{\includegraphics[width=0.95\textwidth] {figures/classdiagram}}
\caption{{\bf Class Diagram of a framework for explosions}.
}
\label{fig:classdiagram}
\figline
\end{figure*}

The animation in this directory contains two classes, derived from
{\sf Sprite}: {\sf ExplodingSprite} and {\sf Explosion}.
The class {\sf ExplodingSprite} uses the SDL surface of the
sprite frame to be exploded and partitions the surface into small 
rectangles or chunks. There is a constant in 
{\sf xmlSpec/game.xml}, {\sf orbChunkSize}; the number of chunks
that will be created is {\sf orbChunkSize}x{\sf orbChunkSize}.

The chunks are placed in a vector and each chunk is treated as a 
sprite in the animation, with each chunk having its own direction 
and velocity. 


\end{document}
